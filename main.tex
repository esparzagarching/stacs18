\documentclass[a4paper,UKenglish]{lipics-v2016}
%This is a template for producing LIPIcs articles. 
%See lipics-manual.pdf for further information.
%for A4 paper format use option "a4paper", for US-letter use option "letterpaper"
%for british hyphenation rules use option "UKenglish", for american hyphenation rules use option "USenglish"
% for section-numbered lemmas etc., use "numberwithinsect"
 
\usepackage{microtype}%if unwanted, comment out or use option "draft"

%\graphicspath{{./graphics/}}%helpful if your graphic files are in another directory

\bibliographystyle{plainurl}% the recommended bibstyle

\usepackage{bm}
\usepackage{macros}

%% Temporary
\usepackage{todonotes}
\newcommand{\alert}[2][]{\todo[color=red, #1]{\tiny MB: \scriptsize #2}}
\newcommand{\javier}[2][]{\todo[color=orange, #1]{\tiny MB: \scriptsize #2}}
\newcommand{\michael}[2][]{\todo[color=cyan, #1]{\tiny JE: \scriptsize #2}}
\newcommand{\stefan}[2][]{\todo[color=green, #1]{\tiny SJ: \scriptsize #2}}

% Author macros::begin %%%%%%%%%%%%%%%%%%%%%%%%%%%%%%%%%%%%%%%%%%%%%%%%
\title{Large Flocks of Small Birds\footnote{M. Blondin was supported by the Fonds de recherche du Quebec – Nature et technologies (FRQNT).}}
% \titlerunning{} %optional, in case that the title is too long; the running title should fit into the top page column

%% Please provide for each author the \author and \affil macro, even when authors have the same affiliation, i.e. for each author there needs to be the  \author and \affil macros
\author[1]{Michael Blondin}
\author[2]{Javier Esparza}
\author[3]{Stefan Jaax}
\affil[1]{Technische Universität München, Munich, Germany\\
  \texttt{blondin@in.tum.de}}
\affil[2]{Technische Universität München, Munich, Germany\\
  \texttt{esparza@in.tum.de}}
\affil[3]{Technische Universität München, Munich, Germany\\
  \texttt{jaax@in.tum.de}}
\authorrunning{M. Blondin and J. Esparza and S. Jaax} %mandatory. First: Use abbreviated first/middle names. Second (only in severe cases): Use first author plus 'et. al.'

\Copyright{Michael Blondin, Javier Esparza and Stefan Jaax}%mandatory, please use full first names. LIPIcs license is "CC-BY";  http://creativecommons.org/licenses/by/3.0/

\subjclass{F.1.1 Models of Computation}% mandatory: Please choose ACM 1998 classifications from http://www.acm.org/about/class/ccs98-html . E.g., cite as "F.1.1 Models of Computation". 
\keywords{Population protocols, Presburger arithmetic}% mandatory: Please provide 1-5 keywords
% Author macros::end %%%%%%%%%%%%%%%%%%%%%%%%%%%%%%%%%%%%%%%%%%%%%%%%%

%Editor-only macros:: begin (do not touch as author)%%%%%%%%%%%%%%%%%%%%%%%%%%%%%%%%%%
\EventEditors{John Q. Open and Joan R. Acces}
\EventNoEds{2}
\EventLongTitle{42nd Conference on Very Important Topics (CVIT 2016)}
\EventShortTitle{CVIT 2016}
\EventAcronym{CVIT}
\EventYear{2016}
\EventDate{December 24--27, 2016}
\EventLocation{Little Whinging, United Kingdom}
\EventLogo{}
\SeriesVolume{42}
\ArticleNo{23}
% Editor-only macros::end %%%%%%%%%%%%%%%%%%%%%%%%%%%%%%%%%%%%%%%%%%%%%%%

\begin{document}

\maketitle

\begin{abstract}
  Lorem ipsum dolor sit amet, consectetur adipiscing elit. Suspendisse
  blandit ante at feugiat posuere. Nunc facilisis ante sed auctor
  blandit. Praesent efficitur ipsum mauris, nec facilisis ex suscipit
  sit amet. Nulla mi nunc, malesuada non pharetra at, interdum tempor
  ante. Ut eleifend lacus lobortis lectus tincidunt gravida. Donec
  massa ligula, euismod sit amet enim quis, maximus mollis odio. Etiam
  elementum pharetra urna, a auctor velit dictum vel. Cras a auctor
  nisl. Sed orci leo, faucibus at ipsum eget, aliquet blandit
  erat. Vivamus id odio elit. Maecenas orci justo, vestibulum sit amet
  purus eget, elementum euismod dolor. Suspendisse sed dignissim
  urna. Integer blandit lacinia diam a fringilla.
\end{abstract}

\section{Introduction}
Lorem ipsum dolor sit amet, consectetur adipiscing elit. Quisque
maximus neque a blandit ultrices. Vestibulum ornare finibus metus nec
hendrerit. Cras nec eros ut elit mattis hendrerit. Suspendisse
vestibulum diam sit amet neque elementum, ut luctus est
tristique. Mauris sollicitudin tincidunt quam id sagittis. Mauris
fringilla commodo mi in luctus. Etiam id sem at ipsum sodales
dignissim. Praesent volutpat, mauris in eleifend placerat, odio ligula
ullamcorper arcu, et blandit tortor metus a tellus.

Aliquam risus erat, semper quis velit ac, consequat vehicula
quam. Integer molestie, justo id semper viverra, est risus volutpat
mauris, rhoncus porttitor tellus velit dignissim ligula. Fusce arcu
mauris, gravida eget dui at, euismod euismod arcu. Proin cursus
facilisis tellus sollicitudin viverra. Aenean eget justo fermentum,
imperdiet mauris nec, tincidunt nisl. Donec ex turpis, suscipit sit
amet turpis id, suscipit porttitor eros. Cras faucibus, lorem eu
sodales pulvinar, est dui interdum tellus, sed egestas nunc sem cursus
mauris. Aenean malesuada nisi finibus ex porttitor
tristique~\cite{AADFP04, AAE06, AAER07}.


\section{Preliminaries}
\input{sec-preliminaries}

\section{Content}
\input{sec-content}

\section{Conclusion and further work}
\input{sec-conclusion}

%% Acknowledgements
\subparagraph*{Acknowledgements.}

We wish to thank...

%% Bibliography
\bibliography{references}

%% Appendix
\clearpage
\appendix
\section{Typesetting instructions -- please read carefully}

Please comply with the following instructions when preparing your
article for a LIPIcs proceedings volume.
\begin{itemize}
\item Use pdflatex and an up-to-date LaTeX system.
\item Use further LaTeX packages only if required. Avoid usage of packages like \verb+enumitem+, \verb+enumerate+, \verb+cleverref+. Keep it simple, i.e. use as few additional packages as possible.
\item Add custom made macros carefully and only those which are needed in the article (i.e., do not simply add your convolute of macros collected over the years).
\item Do not use a different main font. For example, the usage of the \verb+times+-package is forbidden.
\item Provide full author names (especially with regard to the first name) in the \verb+\author+ macro and in the \verb+\Copyright+ macro.
\item Fill out the \verb+\subjclass+ and \verb+\keywords+ macros. For the \verb+\subjclass+, please refer to the ACM classification at \url{http://www.acm.org/about/class/ccs98-html}.
\item Take care of suitable linebreaks and pagebreaks. No overfull \verb+\hboxes+ should occur in the warnings log.
\item Provide suitable graphics of at least 300dpi (preferrably in pdf format).
\item Use the provided sectioning macros: \verb+\section+, \verb+\subsection+, \verb+\subsection*+, \verb+\paragraph+, \verb+\subparagraph*+, ... ``Self-made'' sectioning commands (for example, \verb+\noindent{\bf My+ \verb+subparagraph.}+ will be removed and replaced by standard LIPIcs style sectioning commands.
\item Do not alter the spacing of the  \verb+lipics-v2016.cls+ style file. Such modifications will be removed.
\item Do not use conditional structures to include/exclude content. Instead, please provide only the content that should be published -- in one file -- and nothing else.
\item Remove all comments, especially avoid commenting large text blocks and using \verb+\iffalse+ $\ldots$ \verb+\fi+ constructions.
\item Keep the standard style (\verb+plainurl+) for the bibliography as provided by the\linebreak \verb+lipics-v2016.cls+ style file.
\item Use BibTex and provide exactly one BibTex file for your article. The BibTex file should contain only entries that are referenced in the article. Please make sure that there are no errors and warnings with the referenced BibTex entries.
\item Use a spellchecker to get rid of typos.
\item A manual for the LIPIcs style is available at \url{http://drops.dagstuhl.de/styles/lipics-v2016/lipics-v2016-authors/lipics-v2016-manual.pdf}.
\end{itemize}


\clearpage
\section{Javier's stuff}
\javier{Temporary section for Javier's stuff}


\section{Michael's stuff}
\newcommand{\pn}{\mathcal{N}}
\newcommand{\norm}[1]{\lVert#1\rVert}
\newcommand{\pre}[1]{\ensuremath{{^\bullet #1}}}
\newcommand{\post}[1]{\ensuremath{{#1^\bullet}}}
\newcommand{\prepost}[1]{\ensuremath{{^\bullet {#1} ^\bullet}}}

\michael{Adapt the theorem/proof to Petri nets with weights.}
\begin{theorem}
  Let $\pn = (P, T, F)$ be a Petri net, let $\vec{x}, \vec{y} \in
  \N^P$ and let $\pi \in T^*$ be such that $\vec{x} \trans{\pi}
  \vec{y}$. There exist $\vec{x}', \vec{y}' \in \N^P$ and $S = \{t_1,
  t_2, \ldots, t_n\} \subseteq \supp{\pi}$ such that
  \begin{itemize}
  \item $\vec{x}' \trans{t_1^* t_2^* \cdots t_n^*} \vec{y}'$,

  \item $\norm{\vec{x}'} \leq 2^n$,

  \item $\supp{\vec{x}'} = \supp{\vec{x}}$, and 

  \item $\supp{\vec{y}'} = \supp{\vec{x}} \cup \post{S}$.
  \end{itemize}
\end{theorem}

\begin{proof}
  Let $\vec{x}, \vec{y} \in \N^P$ and $\pi \in T^*$ be such that
  $\vec{x} \trans{\pi} \vec{y}$. We prove the claim by induction on
  $|\pi|$. If $|\pi| = 0$, then $\supp{\pi} = \emptyset$ and $\vec{x}
  = \vec{y}$. Thus, the claim is satisfied by $S \defeq \emptyset$ and
  the markings $\vec{x}'$ and $\vec{y}'$ such that for every $p \in P$,
  $$
  \vec{x}'(p) \defeq \vec{y}'(p) \defeq
  \begin{cases}
    1 & \text{if } p \in \supp{\vec{x}}, \\
    0 & \text{otherwise}.
  \end{cases}
  $$ Assume that $|\pi| > 0$ and that the claim holds for sequences of
  length less than $|\pi|$. There exist $\sigma \in T^*$, $t \in T$
  and $\vec{z} \in \N^P$ such that $\pi = \sigma t$ and $\vec{x}
  \trans{\sigma} \vec{z} \trans{t} \vec{y}$. By induction hypothesis,
  there exist $\vec{x}'', \vec{z}'' \in \N^P$ and $U = \{t_1, t_2,
  \ldots, t_m\} \subseteq \supp{\sigma}$ such that
  \begin{itemize}
  \item $\vec{x}'' \trans{t_1^* t_2^* \cdots t_m^*} \vec{z}''$,

  \item $\norm{\vec{x}''} \leq 2^m$,

  \item $\supp{\vec{x}''} = \supp{\vec{x}}$, and 

  \item $\supp{\vec{z}''} = \supp{\vec{x}} \cup \post{U}$.
  \end{itemize}
  
  If $t \in U$, we are done by taking $\vec{x}' \defeq \vec{x}''$,
  $\vec{y}' \defeq \vec{z}''$ and $S \defeq U$. Thus, we may assume
  that $t \not\in U$. Note that $\supp{\vec{z}} \subseteq
  \supp{\vec{x}} \cup \post{U} = \supp{\vec{z}''}$. Thus, since $t$ is
  enabled in $\vec{z}$, it is also enabled in $\vec{z}''$. Moreover,
  we have $$2 \vec{x}'' \trans{t_1^* t_2^* \cdots t_m^*} 2
  \vec{z}''.$$ Therefore, $2 \vec{x}'' \trans{t_1^* t_2^* \cdots
    t_m^*} 2 \vec{z}'' \trans{t} (\vec{z}'' + \vec{w})$ for some
  marking $\vec{w} \in \N^P$.  We claim that we are done by taking
  $\vec{x}' \defeq 2 \vec{x}''$, $\vec{y}' \defeq \vec{z}'' + \vec{w}$
  and $S \defeq U \cup \{t\}$. Let $n = |U|$. The first and third
  properties of the claim clearly hold. The second property holds
  since $$\norm{\vec{x}'} = 2 \cdot \norm{\vec{x}''} \leq 2 \cdot 2^m
  = 2^{m+1} = 2^n.$$ Finally, the fourth proprety holds since
  \begin{align*}
    \supp{\vec{y}'}
    &= \supp{\vec{z}'' + \vec{w}} \\
    &= \supp{\vec{z}''} \cup \supp{\vec{w}} \\
    &= \supp{\vec{x}} \cup \post{U} \cup \supp{\vec{w}} \\ 
    &= \supp{\vec{x}} \cup \post{U} \cup \post{t}. \\ 
    &= \supp{\vec{x}} \cup \post{S}.\qedhere
  \end{align*}
\end{proof}


\section{Stefan's stuff}
\stefan{Temporary section for Stefan's stuff}


\end{document}
