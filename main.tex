\documentclass[a4paper,UKenglish]{lipics-v2016}
%This is a template for producing LIPIcs articles. 
%See lipics-manual.pdf for further information.
%for A4 paper format use option "a4paper", for US-letter use option "letterpaper"
%for british hyphenation rules use option "UKenglish", for american hyphenation rules use option "USenglish"
% for section-numbered lemmas etc., use "numberwithinsect"
 
\usepackage{microtype}%if unwanted, comment out or use option "draft"

%\graphicspath{{./graphics/}}%helpful if your graphic files are in another directory

\bibliographystyle{plainurl}% the recommended bibstyle

%% General notation
\newcommand{\defeq}{\stackrel{\scriptscriptstyle\text{def}}{=}}
\newcommand{\defiff}{\stackrel{\scriptscriptstyle\text{def}}{\iff}}
\newcommand{\ie}{\text{i.e.}\xspace}
\newcommand{\eg}{\text{e.g.}\xspace}
\newcommand{\etal}{\text{\emph{et al.}}\xspace}

%% Sets and vectors
\newcommand{\N}{\mathbb{N}}                    %% Natural numbers
\newcommand{\Z}{\mathbb{Z}}                    %% Integers
\newcommand{\Q}{\mathbb{Q}}                    %% Rationals
\renewcommand{\vec}[1]{\bm{#1}}                %% Vectors
\newcommand{\set}[1]{\left\{#1\right\}}        %% Custom set notation

%% Multisets
\newcommand{\multiset}[1]{\Lbag#1\Rbag}        %% Bag notation
\newcommand{\mplus}{\mathbin{+}}               %% Multiset addition
\newcommand{\mminus}{\mathbin{\varominus}}     %% Multiset difference
\newcommand{\supp}[1]{\llbracket#1\rrbracket}  %% Support

%% Population protocols notation
\newcommand{\pop}[1]{\mathrm{Pop}(#1)}         %% Set of populations
\newcommand{\PP}{\mathcal{P}}                  %% Population protocol name
\newcommand{\trans}[1]{\xrightarrow{#1}}       %% Transition

%% Temporary
\usepackage{todonotes}
\newcommand{\alert}[2][]{\todo[color=red, #1]{\tiny MB: \scriptsize #2}}
\newcommand{\javier}[2][]{\todo[color=orange, #1]{\tiny MB: \scriptsize #2}}
\newcommand{\michael}[2][]{\todo[color=cyan, #1]{\tiny JE: \scriptsize #2}}
\newcommand{\stefan}[2][]{\todo[color=green, #1]{\tiny SJ: \scriptsize #2}}

% Author macros::begin %%%%%%%%%%%%%%%%%%%%%%%%%%%%%%%%%%%%%%%%%%%%%%%%
\title{Large Flocks of Small Birds\footnote{M. Blondin was supported by the Fonds de recherche du Quebec – Nature et technologies (FRQNT).}}
% \titlerunning{} %optional, in case that the title is too long; the running title should fit into the top page column

%% Please provide for each author the \author and \affil macro, even when authors have the same affiliation, i.e. for each author there needs to be the  \author and \affil macros
\author[1]{Michael Blondin}
\author[2]{Javier Esparza}
\author[3]{Stefan Jaax}
\affil[1]{Technische Universität München, Munich, Germany\\
  \texttt{blondin@in.tum.de}}
\affil[2]{Technische Universität München, Munich, Germany\\
  \texttt{esparza@in.tum.de}}
\affil[3]{Technische Universität München, Munich, Germany\\
  \texttt{jaax@in.tum.de}}
\authorrunning{M. Blondin and J. Esparza and S. Jaax} %mandatory. First: Use abbreviated first/middle names. Second (only in severe cases): Use first author plus 'et. al.'

\Copyright{Michael Blondin, Javier Esparza and Stefan Jaax}%mandatory, please use full first names. LIPIcs license is "CC-BY";  http://creativecommons.org/licenses/by/3.0/

\subjclass{F.1.1 Models of Computation}% mandatory: Please choose ACM 1998 classifications from http://www.acm.org/about/class/ccs98-html . E.g., cite as "F.1.1 Models of Computation". 
\keywords{Population protocols, Presburger arithmetic}% mandatory: Please provide 1-5 keywords
% Author macros::end %%%%%%%%%%%%%%%%%%%%%%%%%%%%%%%%%%%%%%%%%%%%%%%%%

%Editor-only macros:: begin (do not touch as author)%%%%%%%%%%%%%%%%%%%%%%%%%%%%%%%%%%
\EventEditors{John Q. Open and Joan R. Acces}
\EventNoEds{2}
\EventLongTitle{42nd Conference on Very Important Topics (CVIT 2016)}
\EventShortTitle{CVIT 2016}
\EventAcronym{CVIT}
\EventYear{2016}
\EventDate{December 24--27, 2016}
\EventLocation{Little Whinging, United Kingdom}
\EventLogo{}
\SeriesVolume{42}
\ArticleNo{23}
% Editor-only macros::end %%%%%%%%%%%%%%%%%%%%%%%%%%%%%%%%%%%%%%%%%%%%%%%

\begin{document}

\maketitle

\begin{abstract}
  Lorem ipsum dolor sit amet, consectetur adipiscing elit. Suspendisse
  blandit ante at feugiat posuere. Nunc facilisis ante sed auctor
  blandit. Praesent efficitur ipsum mauris, nec facilisis ex suscipit
  sit amet. Nulla mi nunc, malesuada non pharetra at, interdum tempor
  ante. Ut eleifend lacus lobortis lectus tincidunt gravida. Donec
  massa ligula, euismod sit amet enim quis, maximus mollis odio. Etiam
  elementum pharetra urna, a auctor velit dictum vel. Cras a auctor
  nisl. Sed orci leo, faucibus at ipsum eget, aliquet blandit
  erat. Vivamus id odio elit. Maecenas orci justo, vestibulum sit amet
  purus eget, elementum euismod dolor. Suspendisse sed dignissim
  urna. Integer blandit lacinia diam a fringilla.
\end{abstract}

\section{Introduction}
Population protocols~\cite{AADFP04} are a model of distributed
computation by anonymous, identical finite-state agents.  Initially
introduced to model networks of passively mobile sensors
in~\cite{AADFP04}, they also capture the essence of distributed
computation in trust propagation or chemical reactions under the name
of Chemical Reaction Networks, a model essentially equivalent to
population protocols.

Since the agents executing a protocol are anonymous and identical, its
global state---called a \emph{configuration}---is completely
determined by the number of agents at each local state. In each
computation step, a pair of agents, chosen by an adversary subject to
a fairness condition stating that any continuously reachable
configuration is eventually reached, interact and move to new states
according to a joint transition function. In a closely related model,
the adversary chooses the pair of agents uniformly at random.

A protocol computes a boolean value for a given initial configuration
if in all fair executions all agents eventually agree to this
value---so, intuitively, population protocols compute by reaching
consensus. Given a set of initial configurations, the predicate
computed by a protocol is the function that assigns to each
configuration $C$ the boolean value computed by the protocol starting
from $C$.

%% If the protocol does not reach consensus for some input, then we
%% say it is ill specified and does not compute any predicate.

Most research on population protocols has focused on their expressive
power and their speed.  In a famous result, Angluin et al. have shown
that predicates computable by population protocols are exactly the
Presburger predicates, i.e., the predicates definable in Presburger
arithmetic~\cite{AAER07}. The speed of a protocol, defined for the
model with a randomized adversary, is the expected number of pairwise
interactions until all agents have the correct output
value. In~\cite{AAE08a} Angluin et al. showed that for every
Presburger predicate $P$ there is a population protocol with a leader
(essentially, protocols with one distinguished agent having a
different set of states than the rest) that computes $P$ in
$O(n \log^4 n)$ interactions in expectation, where $n$ is the number
of agents of the initial configuration. Many other results provide
lower and upper bounds for specific tasks, like electing a
leader~\cite{DS15}, or specific protocols, like majority~\cite{AGV15}.

In this paper we initiate the study of a third natural question: Given
a protocol, which is the size of the smallest protocol that computes
it? To the best of our knowledge, this problem has not been
investigated so far (see the section below on related work).

In order to introduce our results in the simplest possible setting, in
the first part of the paper we concentrate on the size of protocols
for the family of predicates $\{x \geq 0, x \geq 1, x \geq
2 \}$. These protocols solve the well-known flock-of-birds problem, in
which tiny sensors placed in birds have to reach consensus on whether
the number of sick birds on the flock exceeds a given constant $c$.

\medskip\noindent \textbf{Protocol size for the flock-of-birds
  problem.} The standard protocol for the predicate $x \geq c$ is
presented in many papers as a simple introductpry example. It has
states $\{0, 1, \ldots, c\}$; states $\{0 , \ldots, c-1\}$ have output
$0$, and state $c$ having output 1. Agents start at state $1$. The
interactions ensure that all agents eventually reach state $c$ if{}f
$x \leq c$ holds. They are $(x, y) \mapsto (\min\{x+y,c\}, 0)$ and
$(c,y) \mapsto (c,c)$ for every $0 \leq x, y \leq c$.

In a first, warm-up fase we exhibit a family of protocols which only
have $O(\log c)$ states. More precisely, we prove:
\begin{itemize}
\item[(1)] There exists a family $\mathcal{F} = \{\PP_0, \PP_1,
  \ldots, \}$ of population protocols such that $\PP_i$ has $O(\log_2
  c)$ states and computes the predicate $x \geq i$.
\end{itemize}
We also give a lower bound:
\begin{itemize}
\item[(2)] For every $i \in \N$ and for every family $\mathcal{F} =
  \{\PP_0, \PP_1, \ldots, \PP_c\}$ of population protocols: If $\PP_i$
  computes the predicate $x \geq i$ for every $0 \leq i \leq c$, then
  there exists $0 \leq i \leq c$ such that $\PP_i$ has more than
  $\sqrt{\log_4 c}$ states.
\end{itemize}
However, this bound is only \emph{existential}: Given a concrete
threshold $c$, it does not provide information on the minimal number
of states of the protocols computing the predicate $x \geq c$
itself. Moreover, since it follows from a simple counting argument
(there are at most $c$ protocols with $\sqrt{\log_4 c}$ states), it
does not provide any information on which is the index $i$ realizing
the bound. Can we obtain universal bounds?

We present universal bounds for \emph{consensus-aware} protocols. Both
the standard protocol with $c+1$ states for the predicate $x \geq c$
nd the family $\mathcal{F}$ described above have the following
property: If the number of agents is greater than or equal to $c$,
then the agents not only eventually reach consensus 1: they also
eventually \emph{know} that they will reach this consensus. For
example, it is easy to see that in the standard protocol any agent
that reaches state $c$ knows that eventually all agents will reach it
too. We say that these protocols are consensus-aware.

We first prove that for \emph{leaderless} consensus-aware protocols
there is a universal bound that essentially matches the upper bound
given in (1):
\begin{itemize}
\item[(3)] Every leaderless, consensus-aware population protocol that
  computes the predicate $x \geq c$ has at least $\log_3 c$ states.
\end{itemize}
Can this bound be extended to protocols with a leader? Our most
surprising results prove that this is not the case: For certain values
of $c$ there are protocols with $O(\log \log c)$ states. Further, this
upper bound is close to optimal:
\begin{itemize}
\item[(4)]There exists a family $\mathcal{F} = \{\PP_n : n \in \N\}$
  of consensus-aware population protocols with leader, and values
  $c_0, c_1, \ldots \in \N$, such that $\PP_n$ has $O(\log\log c_n)$
  states and computes the predicate $x \geq c_n$ for every $n \in \N$.
\item[(5)] Every consensus-aware protocol (leaderless or not) that
  computes the predicate $x \geq c$ has at least $\sqrt{\log \log(c /
    2)}$ states.
\end{itemize}

\medskip\noindent \textbf{Protocols for quantifier-free Presburger
  arithmetic.} In the second half of the paper we present results
valid for any predicate expressible in quantifier-free Presburger
arithmetic. The size of a formula depends on the size of the numbers
occurring in it, and on its ``skeleton'', the length of the formula
when numbers are replaced by formal parameters. We give protocols
whose number of states is a polylogarithmic function of the numbers
appearing in a formula. The key result is a subtle protocol for
threshold predicates of the form $\sum_{i=1}^n a_i x_i \geq c$, where
$a_i \in \Z$ for every $1 \leq i \leq n$ of size GUESS $O(
\sum_{i=1}^n \log|a_i|^{?} )$.

\medskip\noindent \textbf{Related work.} The minimal number of states
has been determined for some specific predicates, like the majority
predicate $x \geq y$, which is proved to require at least 4 states in
\cite{}.  Our results are for families of protocols. There is also
recent work on the trade-off of the speed of a population protocol and
its number of states~\cite{AAEGR17}. However, these results are
achieved for a different model in which the number of states of an
agent is not constant, but a function of the number of agents. In
other words, while in the standard model a protocol with a fixed
number of states computes the predicate for all possible inputs, in
this model the predicate is computed by an infinite failiy of
protocols, one for each possible input.


\medskip\noindent \textbf{Structure of the paper.} In Section~\ref{}
we introduce basic definitions. Section~\ref{} presents a construction
we use thorughout the paper. Section~\ref{} presents our results for
the flock-of-birds predicates. Section~\ref{} sketches how to extend

%% showing that $k$-way protocols, i.e., a generalization of the standard
%% mode in which interactions can involve$2, 3, \ldots, k$ processes can
%% be simulated by the standard 2-way protocols with very low blow-up in
%% the number of states.


\section{Preliminaries}
\input{sec-preliminaries}

\section{Content}
\input{sec-content}

\section{Conclusion and further work}
\input{sec-conclusion}

%% Acknowledgements
\subparagraph*{Acknowledgements.}

We wish to thank...

%% Bibliography
\bibliography{references}

%% Appendix
\clearpage
\appendix
\section{Typesetting instructions -- please read carefully}

Please comply with the following instructions when preparing your
article for a LIPIcs proceedings volume.
\begin{itemize}
\item Use pdflatex and an up-to-date LaTeX system.
\item Use further LaTeX packages only if required. Avoid usage of packages like \verb+enumitem+, \verb+enumerate+, \verb+cleverref+. Keep it simple, i.e. use as few additional packages as possible.
\item Add custom made macros carefully and only those which are needed in the article (i.e., do not simply add your convolute of macros collected over the years).
\item Do not use a different main font. For example, the usage of the \verb+times+-package is forbidden.
\item Provide full author names (especially with regard to the first name) in the \verb+\author+ macro and in the \verb+\Copyright+ macro.
\item Fill out the \verb+\subjclass+ and \verb+\keywords+ macros. For the \verb+\subjclass+, please refer to the ACM classification at \url{http://www.acm.org/about/class/ccs98-html}.
\item Take care of suitable linebreaks and pagebreaks. No overfull \verb+\hboxes+ should occur in the warnings log.
\item Provide suitable graphics of at least 300dpi (preferrably in pdf format).
\item Use the provided sectioning macros: \verb+\section+, \verb+\subsection+, \verb+\subsection*+, \verb+\paragraph+, \verb+\subparagraph*+, ... ``Self-made'' sectioning commands (for example, \verb+\noindent{\bf My+ \verb+subparagraph.}+ will be removed and replaced by standard LIPIcs style sectioning commands.
\item Do not alter the spacing of the  \verb+lipics-v2016.cls+ style file. Such modifications will be removed.
\item Do not use conditional structures to include/exclude content. Instead, please provide only the content that should be published -- in one file -- and nothing else.
\item Remove all comments, especially avoid commenting large text blocks and using \verb+\iffalse+ $\ldots$ \verb+\fi+ constructions.
\item Keep the standard style (\verb+plainurl+) for the bibliography as provided by the\linebreak \verb+lipics-v2016.cls+ style file.
\item Use BibTex and provide exactly one BibTex file for your article. The BibTex file should contain only entries that are referenced in the article. Please make sure that there are no errors and warnings with the referenced BibTex entries.
\item Use a spellchecker to get rid of typos.
\item A manual for the LIPIcs style is available at \url{http://drops.dagstuhl.de/styles/lipics-v2016/lipics-v2016-authors/lipics-v2016-manual.pdf}.
\end{itemize}


\clearpage
\section{Javier's stuff}
\javier{Temporary section for Javier's stuff}


\section{Michael's stuff}
Current notation:
\begin{itemize}
  \item $\norm{\vec{x}}$ and $\norm{W}$ denote respectively the
    largest absolute value that occurs in $\vec{x}$ and $W$;

  \item $\supp{\vec{x}} \defeq \{p \in P : \vec{x}(p) > 0\}$ and
    $\supp{t_1 t_2 \cdots t_n} \defeq \{t_i : 1 \leq i \leq n\}$;

  \item $\pre{p} \defeq \{t \in T : W(t, p) > 0\}$ and $\post{p}
    \defeq \{t \in T : W(p, t) > 0\}$;

  \item $\pre{t} \defeq \{p \in P : W(p, t) > 0\}$ and $\post{t}
    \defeq \{p \in P : W(t, p) > 0\}$.
\end{itemize}
Notions to define:
\begin{itemize}
\item $k$-way protocols;
\item ``agents-know'' protocols;
\item ``leaders''.
\end{itemize}

\begin{proposition}\label{prop:short:saturation}
  Let $\pn = (P, T, W)$ be a Petri net, let $\vec{x}, \vec{y} \in
  \N^P$ and let $\pi \in T^*$ be such that $\vec{x} \trans{\pi}
  \vec{y}$. Let $c \defeq \norm{W} + 1$. There exist $\vec{x}',
  \vec{y}' \in \N^P$ and $S \subseteq \{t_1, t_2, \ldots, t_n\} =
  \supp{\pi}$ such that
  \begin{itemize}
  \item $\vec{x}' \trans{t_1^{c^{n-1}} t_2^{c^{n-2}} \cdots\ t_n}
    \vec{y}'$,

  \item $n \leq |\post{S}|$,

  \item $\norm{\vec{x}'} \leq c^n$,

  \item $\supp{\vec{x}'} = \supp{\vec{x}}$, and 

  \item $\supp{\vec{y}'} = \supp{\vec{x}} \cup \post{S}$.
  \end{itemize}
\end{proposition}

\begin{proof}
  Let $\vec{x}, \vec{y} \in \N^P$ and $\pi \in T^*$ be such that
  $\vec{x} \trans{\pi} \vec{y}$. We prove the claim by induction on
  $|\pi|$. If $|\pi| = 0$, then $\supp{\pi} = \emptyset$ and $\vec{x}
  = \vec{y}$. Thus, the claim is satisfied by $S \defeq \emptyset$ and
  the markings $\vec{x}'$ and $\vec{y}'$ such that for every $p \in
  P$,
  $$
  \vec{x}'(p) \defeq \vec{y}'(p) \defeq
  \begin{cases}
    1 & \text{if } p \in \supp{\vec{x}}, \\
    0 & \text{otherwise}.
  \end{cases}
  $$ Assume that $|\pi| > 0$ and that the claim holds for sequences of
  length less than $|\pi|$. There exist $\sigma \in T^*$, $t \in T$
  and $\vec{z} \in \N^P$ such that $\pi = \sigma t$ and $\vec{x}
  \trans{\sigma} \vec{z} \trans{t} \vec{y}$. By induction hypothesis,
  there exist $\vec{x}'', \vec{z}'' \in \N^P$ and $U = \{t_1, t_2,
  \ldots, t_m\} \subseteq \supp{\sigma}$ such that
  \begin{itemize}
  \item $\vec{x}'' \trans{t_1^{c^{m-1}} t_2^{c^{m-2}} \cdots\ t_m}
    \vec{z}''$,

  \item $m \leq |\post{U}|$,

  \item $\norm{\vec{x}''} \leq c^m$,

  \item $\supp{\vec{x}''} = \supp{\vec{x}}$, and 

  \item $\supp{\vec{z}''} = \supp{\vec{x}} \cup \post{U}$.
  \end{itemize}
  
  If $\post{t} \subseteq \post{U}$, then $\post{S} = \post{U}$ and
  hence we are done by taking $\vec{x}' \defeq \vec{x}''$ and
  $\vec{y}' \defeq \vec{z}''$. Therefore, we may assume that $\post{t}
  \not\subseteq \post{U}$. Note that $\supp{\vec{z}} \subseteq
  \supp{\vec{x}} \cup \post{U} = \supp{\vec{z}''}$. Thus, since $t$ is
  enabled in $\vec{z}$ and since $t$ consumes at most $c - 1$ tokens,
  it is also enabled in $(c - 1) \cdot \vec{z}''$. Moreover, we
  have $$c \cdot \vec{x}'' \trans{t_1^{c^m} t_2^{c^{m-1}} \cdots
    t_m^c} c \cdot \vec{z}''.$$ We obtain $c \cdot \vec{x}''
  \trans{t_1^{c^m} t_2^{c^{m-1}} \cdots t_m^c} c \cdot \vec{z}''
  \trans{t} (\vec{z}'' + \vec{w})$ for some $\vec{w} \in \N^P$.  We
  claim that we are done by taking $S \defeq \{t_1, t_2, \ldots, t_n,
  t\}$, $\vec{x}' \defeq c \cdot \vec{x}''$ and $\vec{y}' \defeq
  \vec{z}'' + \vec{w}$. Let $n \defeq |S|$. The first property of
  the claim holds since
  $$c \cdot \vec{x}'' \trans{t_1^{c^{n-1}} t_2^{c^{n-2}} \cdots t_m^c
    t} \vec{y}'.$$ The second property holds since $n = m + 1 \leq
  |\post{U}| + 1 = |\post{S}|$. The third property holds since
  $\norm{\vec{x}'} = c \cdot \norm{\vec{x}''} \leq c \cdot c^m =
  c^{m+1} = c^n$. The fourth property holds since $\supp{\vec{x}'} =
  \supp{c \cdot \vec{x}''} = \supp{\vec{x}''} =
  \supp{\vec{x}}$. Finally, the fifth proprety holds since
  \begin{align*}
    \supp{\vec{y}'}
    &= \supp{\vec{z}'' + \vec{w}} \\
    &= \supp{\vec{z}''} \cup \supp{\vec{w}} \\
    &= \supp{\vec{x}} \cup \post{U} \cup \supp{\vec{w}} \\ 
    &= \supp{\vec{x}} \cup \post{U} \cup \post{t}. \\ 
    &= \supp{\vec{x}} \cup \post{S}.\qedhere
  \end{align*}
\end{proof}

\begin{corollary}\label{prop:short:cover}
  Let $\pn = (P, T, W)$ be a Petri net, let $\vec{x}, \vec{y} \in
  \N^P$, let $c \defeq \norm{W} + 1$ and let $d \defeq \norm{y}$. If
  $\vec{y}$ is coverable from $\vec{x}$, then $\vec{y}$ is coverable
  from some $\vec{x}' \in \N^P$ such that $\supp{\vec{x}'} =
  \supp{\vec{x}}$ and $\norm{\vec{x}'} \leq d \cdot c^{|P|}$, through
  a sequence $\pi$ of length less than $d \cdot c^{|P|}$.
\end{corollary}

\begin{proof}
  Assume $\vec{y}$ is coverable from $\vec{x}$. There exists some
  marking $\vec{w} \geq \vec{y}$ such that $\vec{x} \trans{*}
  \vec{w}$. By Proposition~\ref{prop:short:saturation}, there exist
  $\vec{v}, \vec{w}' \in \N^P$ and $S = \{t_1, t_2, \ldots, t_n\}
  \subseteq T$ such that
  \begin{itemize}
  \item $\vec{v} \trans{t_1^{c^{n-1}} t_2^{c^{n-2}} \cdots\ t_n}
    \vec{w}'$,

  \item $n \leq |\post{S}| \leq |P|$,

  \item $\norm{\vec{v}} \leq c^n$,

  \item $\supp{\vec{v}} = \supp{\vec{x}}$, and 

  \item $\supp{\vec{w}'} = \supp{\vec{x}} \cup \post{S}$.
  \end{itemize}

  Let $\vec{x}' \defeq d \cdot \vec{v}$, $\vec{y}' \defeq d \cdot
  \vec{w}'$, and $\pi \defeq t_1^{d \cdot c^{n-1}} t_2^{d \cdot
    c^{n-2}} \cdots\ t_n^{d}$. We have $\vec{x}' \trans{\pi}
  \vec{y}'$, $\supp{\vec{x}'} = \supp{\vec{v}} = \supp{\vec{x}}$ and
  $\norm{\vec{x}'} \leq d \cdot c^n$. Moreover, we have $\vec{y} \leq
  \vec{y}'$ since $\supp{\vec{y}} \subseteq \supp{\vec{w}} \subseteq
  \supp{\vec{x}} \cup \post{S} = \supp{\vec{w}'} = \supp{\vec{y}'}$
  and $\vec{y}'(p) \geq d$ for every $p \in \supp{\vec{y}'}$. To
  conclude the proof, observe that $|\pi| < d \cdot c^n$.
\end{proof}

\begin{theorem}
  Let $\PP$ be an ``agents know''-population protocol without
  leader. If $\PP$ computes the predicate $x \geq c$, then $\PP$ has
  at least $\log_3 c$ states.
\end{theorem}

\begin{proof}
  Assume $\PP = (Q, T, \Sigma, I, O)$ computes the predicate $x \geq
  c$. Let $R \subseteq Q$ be the set of ``know-states'' of $\PP$. Let
  $\pn = (Q, T, W)$ be the Petri net associated to $\PP$. There exists
  $q \in R$ such that $\multiset{q}$ is coverable from $\multiset{c
    \cdot x}$ in $\pn$. Note that $\norm{W} \leq 2$. Thus, by
  Corollary~\ref{prop:short:cover}, there exist $\lambda \leq 3^{|Q|}$
  and a marking $\vec{y} \geq \multiset{q}$ such that
  $\multiset{\lambda \cdot x} \trans{*} \vec{y}$. Since $\vec{y}(q) >
  0$, $\PP$ stabilizes to true on input $\multiset{\lambda \cdot
    x}$. This implies that $\lambda \geq c$ and hence that $3^{|Q|}
  \geq c$. Therefore, $|Q| \geq \log_3 c$.
\end{proof}

\begin{theorem}
  Let $\PP$ be an ``agents know''-population protocol with leader. If
  $\PP$ computes the predicate $x \geq c$, then $\PP$ has at least
  $(\log \log(c / 2) / (k \cdot 3^{1 / k}))^{k / (k + 1)}$ states for
  every $k \in \N_{>0}$. \michael{We could also fix $k = 1$ and simply
    have a square root. Depends how precise vs. cryptic we want to
    be.}
\end{theorem}

\begin{proof}
  Assume $\PP = (Q, T, \Sigma, I, O)$ computes the predicate $x \geq
  c$. Let $R \subseteq Q$ be the set of ``know-states'' of $\PP$. Let
  $\pn = (Q, T, W)$ be the Petri net associated to $\PP$. There exist
  $p, q \in Q$ and $r \in R$ such that $\multiset{q, r}$ is coverable
  from $\multiset{p, c \cdot x}$. By~\cite{Rac78}, if a marking
  $\vec{y}$ is coverable from a marking $\vec{x}$, then it is
  coverable from a sequence of length at most $2^{{(3 \cdot |Q| \cdot
      \log \norm{\vec{y}})}^{|Q| \cdot \log
      \norm{\vec{y}}}}$. Therefore, there exist $\pi \in T^*$ and
  $\vec{z} \geq \multiset{q, r}$ such that $\multiset{p, c \cdot x}
  \trans{\pi} \vec{z}$ and $|\pi| \leq 2^{(3 \cdot |Q|)^{|Q|}}$.

  We claim that $c \leq 2 \cdot |\pi|$. For the sake of contradiction,
  assume that $c > 2 \cdot |\pi|$. Note that $\pi$ consumes at most $2
  \cdot |\pi|$ tokens from place $x$. Therefore, $\pi$ is enabled at
  $\multiset{p, (c - 1) \cdot x}$. Let $\vec{z}' \defeq \vec{z} -
  \multiset{x}$. We obtain $\multiset{p, (c-1) \cdot x} \trans{\pi}
  \vec{z}'$. Since $x \not\in R$, we have $\vec{z}'(q) > 0$. Thus,
  $\PP$ stabilizes to true from $\multiset{p, (c-1) \cdot x}$, which
  is a contradiction.

  Therefore, $c \leq 2 \cdot |\pi| \leq 2 \cdot 2^{(3 \cdot
    |Q|)^{|Q|}} = 2 \cdot 2^{2^{|Q| \cdot \log(3 \cdot |Q|)}}$. This
  implies that $\log \log (c / 2) \leq |Q| \cdot \log(3 \cdot
  |Q|)$. Let $k \in \N_{>0}$. Note that $\log(3 \cdot |Q|) \leq k
  \cdot (3 \cdot |Q|)^{1 / k}$. Thus, we have $(\log \log(c / 2) / (k
  \cdot 3^{1 / k}))^{k / (k + 1)} \leq |Q|$.
\end{proof}

\begin{theorem}\label{thm:loglog:upperbound}
  For every $n \in \N$, there exist $c \geq 2^{2^n}$ and a 5-way
  ``agents know''-population protocol with two leaders that has $14n +
  11$ states, has at most $34n + 19$ transitions, and computes the
  predicate $x \geq c$.
\end{theorem}

\begin{proof}
  We build a population protocol based on a construction of Mayr and
  Meyer. Let $n \in \N$. In~\cite[Sect.~6]{MM82}, Mayr and Meyer
  describe a \emph{reversible} Petri net $\pn = (P, T, W)$
  satisfying the following properties:
  \begin{enumerate}
    \item $|P| = 14n + 10$, $|T| = 20n + 8$ and $1 \leq \norm{W} \leq 5$,
    \item $\{q_B, q_C, q_F, q_S\} \subseteq P$, and
    \item $\multiset{q_C, q_S} \trans{*} \vec{y}$ and $\vec{y}(q_F) >
      0$ if and only if $\vec{y} = \multiset{q_C, q_F, q_B^{2^{2^n}}}$
      (by~\cite[Lemma~6 and~8]{MM82}).
  \end{enumerate}

  Let $\PP = (Q, S, \{x\}, I, O)$ be the $5$-way population protocol
  such that
  \begin{align*}
    Q &\defeq P \cup \{q_X\}, \\
    S &\defeq S' \cup \{q_F, r \mapsto q_F, q_F\} \\
    \Sigma &\defeq \{x\}, \\
    I &\defeq \multiset{k \cdot x} \mapsto \multiset{k \cdot q_X,
      q_C, q_S}, \text{ and} \\
    O &\defeq r \mapsto 1 \text{ if } r = q_F \text{ otherwise } 0.
  \end{align*}
  For every $t \in T$, let $\mathrm{pad}(t)$ be the transition
  obtained by ``padding'' $t$ with interactions with $q_X$. For
  example, for a transition $t \in T$ which consumes three tokens from
  $p$, and produces a token into $q$ and $r$, we have $\mathrm{pad}(t)
  = (p, p, p, q_X, q_X) \mapsto (q, r, q_X, q_X, q_X)$. We define the
  remaining transitions of $\PP$ as $S' \defeq \{\mathrm{pad}(t) : t
  \in T, q_F \not\in \pre{t}\}$. Intuitively, $\PP$ behaves as $\pn$,
  except that agents in state $q_F$ remain in $q_F$ and can turn other
  agents into state $q_F$.

  Let us prove that $\PP$ is well-specified. Let $C_0$ be an initial
  configuration of $\PP$. Assume $C_0$ cannot reach a configuration
  $C$ such that $C(q_F) > 0$. As $q_F$ is the only state with output
  $1$, every fair execution from $C_0$ has output $0$. Now, assume
  that $C_0$ can reach such a configuration $C$ and, for the sake of
  contradiction, that there exists a fair execution $C_0 C_1 \cdots$
  such that $C_0(q_F) = C_1(q_F) = \cdots = 0$. Since $\PP$ is
  reversible over $\pop{Q \setminus \{q_F\}}$, we have $C_i \trans{*}
  C_0 \trans{*} C$ for every $i \in \N$. This leads to a contradiction
  since fairness ensures that $C_j = C$ for some $j \in \N$. Thus,
  every fair execution from $C_0$ leads to a configuration that
  contains $q_F$. Since, agents in state $q_F$ cannot be reverted and
  can turn other agents into $q_F$, by fairness, every fair execution
  from $C_0$ stabilizes to $1$.

  For every $k \in \N$, let $D_k = I(\multiset{k \cdot x})$. Let $c
  \in \N$ be the smallest number such that $O(D_c) = 1$. Such a $c$
  exists by~(3). Moreover, we have $c \geq 2^{2^n}$ by~(3). Let $c'
  \geq c$. Since $O(D_c) = 1$, $D_c \trans{*} C$ for some
  configuration such that $C(q_F) > 0$. Note that $D_{c'} \geq
  D_c$. Therefore, by monotonicity, there exists $C'$ such that
  $D_{c'} \trans{*} C'$ and $C'(q_F) > 0$. This implies that
  $O(D_{c'}) = 1$.
\end{proof}

\begin{corollary}
  There exist a family $\mathcal{F} = \{\PP_n : n \in \N\}$ of
  ``agents-know''-population protocols with leaders, and values $c_0,
  c_1, \ldots \in \N$, such that for every $n \in \N$, $\PP_n$ has at
  most $354 \log\log c_n + 201$ states and computes the predicate $x
  \geq c_n$.
\end{corollary}

\begin{proof}
  Let $n \in \N$. By Theorem~\ref{thm:loglog:upperbound}, there exist
  $c \geq 2^{2^n}$ and a 5-way ``agents know''-population protocol
  $\PP = (Q, T, \Sigma, I, O)$ with two leaders such that
  \begin{itemize}
    \item $|Q| = 14n + 11$,
    \item $|T| \leq 34n + 19$,
    \item $c \geq 2^{2^n}$,
    \item $\PP$ computes the predicate $x \geq c$.
  \end{itemize}
  By Lemma~\ref{}, there exists a (2-way) ``agents know''-population
  protocol $\PP' = (Q', T', \Sigma, I', O')$ with two leaders such
  that
  \begin{itemize}
  \item $|Q'| = |Q| + 10 \cdot |T| \leq (14n + 11) + (340n + 190) =
    354n + 201$,
  \item $|T'| = 7 \cdot |T| \leq 238n + 133$ and
  \item $\PP'$ computes the predicate $x \geq c$.
  \end{itemize}
  We take $\PP_n \defeq \PP'$ and $c_n \defeq c$. We are done since
  $|Q'| \leq 354n + 201 \leq 354 \log\log c + 201$.
\end{proof}

\begin{theorem}
  Let $\mathcal{F} = \{\PP_0, \PP_1, \ldots, \PP_c\}$ be a finite
  family of population protocols such that $\PP_i$ computes the
  predicate $x \geq i$ for every $0 \leq i \leq c$. There exists $0
  \leq i \leq c$ such that $\PP_i$ has more than $\sqrt{\log_4 c}$
  states.
\end{theorem}

\begin{proof}
  For the sake of contradiction, assume that every protocol of
  $\mathcal{F}$ has at most $\sqrt{\log_4 c}$ states. Let $d(m)$ be
  the number of distinct population protocols with at most $m$
  states. We have:
  \begin{align*}
    d(m)
    &\leq \sum_{i = 1}^m i \cdot 2^{i \choose 2} \cdot i \cdot 2^i &
    (\text{\# states} \cdot \text{\# transitions} \cdot \text{\# input
      map.} \cdot \text{\# output map.}) \\
    &\leq m \cdot 2^{m \choose 2} \cdot m \cdot 2^m \\
    &= m^2 \cdot 2^{m(m - 1)/2 + m} \\
    &\leq m^2 \cdot 2^{m^2} \\
    &\leq 2^{2m^2} \\
    &= 4^{m^2}.
  \end{align*}
  Therefore, $d(\sqrt{\log_4 c}) \leq c < |\mathcal{F}|$ which is a
  contradiction.
\end{proof}


\section{Stefan's stuff}
\stefan{Temporary section for Stefan's stuff}


\end{document}
