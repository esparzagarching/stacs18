Population protocols~\cite{AADFP04} are a model of distributed
computation by anonymous, identical finite-state agents.  Initially
introduced to model networks of passively mobile sensors
in~\cite{AADFP04}, they also capture the essence of distributed
computation in trust propagation or chemical reactions under the name
of Chemical Reaction Networks, a model essentially equivalent to
population protocols.

Since the agents executing a protocol are anonymous and identical, its
global state---called a \emph{configuration}---is completely
determined by the number of agents at each local state. In each
computation step, a pair of agents, chosen by an adversary subject to
a fairness condition stating that any continuously reachable
configuration is eventually reached, interact and move to new states
according to a joint transition function. In a closely related model,
the adversary chooses the pair of agents uniformly at random.

A protocol computes a boolean value for a given initial configuration
if in all fair executions all agents eventually agree to this
value---so, intuitively, population protocols compute by reaching
consensus. Given a set of initial configurations, the predicate
computed by a protocol is the function that assigns to each
configuration $C$ the boolean value computed by the protocol starting
from $C$.

%% If the protocol does not reach consensus for some input, then we
%% say it is ill specified and does not compute any predicate.

Most research on population protocols has focused on their expressive
power and their speed.  In a famous result, Angluin et al.\ have shown
that predicates computable by population protocols are exactly the
Presburger predicates, i.e., the predicates definable in Presburger
arithmetic~\cite{AAER07}. The speed of a protocol, defined for the
model with a randomized adversary, is the expected number of pairwise
interactions until all agents have the correct output
value. In~\cite{AAE08a}, Angluin et al.\ showed that for every
Presburger predicate $P$ there is a population protocol with a leader
(essentially, protocols with one distinguished agent having a
different set of states than the rest) that computes $P$ in
$O(n \log^4 n)$ interactions in expectation, where $n$ is the number
of agents of the initial configuration. Many other results provide
lower and upper bounds for specific tasks, like electing a
leader~\cite{DS15}, or specific protocols, like majority~\cite{AGV15}.

In this paper we initiate the study of a third natural question: Given
a protocol, which is the size of the smallest protocol that computes
it? To the best of our knowledge, this problem has not been
investigated so far (see the section below on related work).

In order to introduce our results in the simplest possible setting, in
the first part of the paper we concentrate on the size of protocols
for the family of predicates $\{x \geq 0, x \geq 1, x \geq
2 \}$. These protocols solve the well-known flock-of-birds problem, in
which tiny sensors placed in birds have to reach consensus on whether
the number of sick birds on the flock exceeds a given constant $c$.

\medskip\noindent \textbf{Protocol size for the flock-of-birds
  problem.} The standard protocol for the predicate $x \geq c$ is
presented in many papers as a simple introductory example. It has
states $\{0, 1, \ldots, c\}$; states $\{0 , \ldots, c-1\}$ have output
$0$, and state $c$ having output 1. Agents start at state $1$. The
interactions ensure that all agents eventually reach state $c$ if{}f
$x \leq c$ holds. They are $(x, y) \mapsto (\min\{x+y,c\}, 0)$ and
$(c,y) \mapsto (c,c)$ for every $0 \leq x, y \leq c$.

In a first, warm-up phase we exhibit a family of protocols which only
have $O(\log c)$ states. More precisely, we prove:
\begin{itemize}
\item[(1)] There exists a family $\mathcal{F} = \{\PP_0, \PP_1,
  \ldots, \}$ of population protocols such that $\PP_i$ has $O(\log_2
  c)$ states and computes the predicate $x \geq i$.
\end{itemize}
We also give a lower bound:
\begin{itemize}
\item[(2)] For every $i \in \N$ and for every family $\mathcal{F} =
  \{\PP_0, \PP_1, \ldots, \PP_c\}$ of population protocols: If $\PP_i$
  computes the predicate $x \geq i$ for every $0 \leq i \leq c$, then
  there exists $0 \leq i \leq c$ such that $\PP_i$ has more than
  $\sqrt{\log_4 c}$ states.
\end{itemize}
However, this bound is only \emph{existential}: Given a concrete
threshold $c$, it does not provide information on the minimal number
of states of the protocols computing the predicate $x \geq c$
itself. Moreover, since it follows from a simple counting argument
(there are at most $c$ protocols with $\sqrt{\log_4 c}$ states), it
does not provide any information on which is the index $i$ realizing
the bound. Can we obtain universal bounds?

We present universal bounds for \emph{consensus-aware} protocols. Both
the standard protocol with $c+1$ states for the predicate $x \geq c$
nd the family $\mathcal{F}$ described above have the following
property: If the number of agents is greater than or equal to $c$,
then the agents not only eventually reach consensus 1: they also
eventually \emph{know} that they will reach this consensus. For
example, it is easy to see that in the standard protocol any agent
that reaches state $c$ knows that eventually all agents will reach it
too. We say that these protocols are consensus-aware.

We first prove that for \emph{leaderless} consensus-aware protocols
there is a universal bound that essentially matches the upper bound
given in (1):
\begin{itemize}
\item[(3)] Every leaderless, consensus-aware population protocol that
  computes the predicate $x \geq c$ has at least $\log_3 c$ states.
\end{itemize}
Can this bound be extended to protocols with a leader? Our most
surprising results prove that this is not the case: For certain values
of $c$ there are protocols with $O(\log \log c)$ states. Further, this
upper bound is close to optimal:
\begin{itemize}
\item[(4)]There exists a family $\mathcal{F} = \{\PP_n : n \in \N\}$
  of consensus-aware population protocols with leader, and values
  $c_0, c_1, \ldots \in \N$, such that $\PP_n$ has $O(\log\log c_n)$
  states and computes the predicate $x \geq c_n$ for every $n \in \N$.
\item[(5)] Every consensus-aware protocol (leaderless or not) that
  computes the predicate $x \geq c$ has at least $\sqrt{\log \log(c /
    2)}$ states.
\end{itemize}

\medskip\noindent \textbf{Protocols for quantifier-free Presburger
  arithmetic.} In the second half of the paper we present results
valid for any predicate expressible in quantifier-free Presburger
arithmetic. The size of a formula depends on the size of the numbers
occurring in it, and on its ``skeleton'', the length of the formula
when numbers are replaced by formal parameters. We give protocols
whose number of states is a polylogarithmic function of the numbers
appearing in a formula. The key result is a subtle protocol for
threshold predicates of the form $\sum_{i=1}^n a_i x_i \geq c$, where
$a_i \in \Z$ for every $1 \leq i \leq n$ of size GUESS $O(
\sum_{i=1}^n \log|a_i|^{?} )$.

\medskip\noindent \textbf{Related work.} The minimal number of states
has been determined for some specific predicates, like the majority
predicate $x \geq y$, which is proved to require at least 4 states in
\cite{}.  Our results are for families of protocols. There is also
recent work on the trade-off of the speed of a population protocol and
its number of states~\cite{AAEGR17}. However, these results are
achieved for a different model in which the number of states of an
agent is not constant, but a function of the number of agents. In
other words, while in the standard model a protocol with a fixed
number of states computes the predicate for all possible inputs, in
this model the predicate is computed by an infinite familiy of
protocols, one for each possible input.


\medskip\noindent \textbf{Structure of the paper.} In Section~\ref{}
we introduce basic definitions. Section~\ref{} presents a construction
we use thorughout the paper. Section~\ref{} presents our results for
the flock-of-birds predicates. Section~\ref{} sketches how to extend...

%% showing that $k$-way protocols, i.e., a generalization of the standard
%% mode in which interactions can involve$2, 3, \ldots, k$ processes can
%% be simulated by the standard 2-way protocols with very low blow-up in
%% the number of states.
